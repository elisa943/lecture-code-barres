%----------------------------------------------------------------------------------------
%  PACKAGES AND CONFIGURATION
%----------------------------------------------------------------------------------------

\documentclass{rapport}
\usepackage{geometry}
\usepackage{fancyhdr} % For custom headers
\usepackage{lastpage} % To determine the last page for the footer
\usepackage{float}
\usepackage{extramarks} % For headers and footers
\usepackage[most]{tcolorbox} % For problem answer sections
\usepackage{graphicx} % For inserting images
\usepackage{xcolor} % For link coloring
\usepackage[hidelinks]{hyperref} % For URL links (no box or color name) 

%Ce que moi(Alban) ai rajouté comme package
\usepackage{amsfonts} % For math sign like Z
\usepackage{amsmath} % For \genfrac
\usepackage{amssymb} % pour \lessgtr
\usepackage{comment} 
\usepackage{multicol} % Pour les colonnes
\usepackage{listings}
\usepackage{courier}

\definecolor{backcolour}{RGB}{47,47,47}
\definecolor{violet}{RGB}{200,20,200}
\definecolor{codegreen}{rgb}{0,0.6,0}
\definecolor{codegray}{rgb}{0.5,0.5,0.5}
\definecolor{marron}{RGB}{150,100,70}
\definecolor{red2}{RGB}{207,43,58}
\definecolor{bordeaux}{RGB}{74,31,30}
\lstdefinestyle{mystyle}{
    language=Python,
    basicstyle=\tiny\ttfamily
    backgroundcolor=\color{backcolour},   
    commentstyle=\color{codegreen},
    keywordstyle=\color{violet},
    numberstyle=\tiny\color{codegray},
    stringstyle=\color{red},
    literate={,}{{\textcolor{red}{,}}}1
             {;}{{\textcolor{red}{;}}}1,
    basicstyle=\ttfamily\footnotesize,
    breakatwhitespace=false,         
    breaklines=true,                 
    captionpos=b,                    
    keepspaces=true,                 
    numbers=left,                    
    numbersep=5pt,                  
    showspaces=false,                
    showstringspaces=false,
    showtabs=false,                  
    tabsize=2,
    literate=
  {á}{{\'a}}1 {é}{{\'e}}1 {í}{{\'i}}1 {ó}{{\'o}}1 {ú}{{\'u}}1
  {Á}{{\'A}}1 {É}{{\'E}}1 {Í}{{\'I}}1 {Ó}{{\'O}}1 {Ú}{{\'U}}1
  {à}{{\`a}}1 {è}{{\`e}}1 {ì}{{\`i}}1 {ò}{{\`o}}1 {ù}{{\`u}}1
  {À}{{\`A}}1 {È}{{\'E}}1 {Ì}{{\`I}}1 {Ò}{{\`O}}1 {Ù}{{\`U}}1
  {ä}{{\"a}}1 {ë}{{\"e}}1 {ï}{{\"i}}1 {ö}{{\"o}}1 {ü}{{\"u}}1
  {Ä}{{\"A}}1 {Ë}{{\"E}}1 {Ï}{{\"I}}1 {Ö}{{\"O}}1 {Ü}{{\"U}}1
  {â}{{\^a}}1 {ê}{{\^e}}1 {î}{{\^i}}1 {ô}{{\^o}}1 {û}{{\^u}}1
  {Â}{{\^A}}1 {Ê}{{\^E}}1 {Î}{{\^I}}1 {Ô}{{\^O}}1 {Û}{{\^U}}1
  {œ}{{\oe}}1 {Œ}{{\OE}}1 {æ}{{\ae}}1 {Æ}{{\AE}}1 {ß}{{\ss}}1
  {ű}{{\H{u}}}1 {Ű}{{\H{U}}}1 {ő}{{\H{o}}}1 {Ő}{{\H{O}}}1
  {ç}{{\c c}}1 {Ç}{{\c C}}1 {ø}{{\o}}1 {å}{{\r a}}1 {Å}{{\r A}}1
  {€}{{\EUR}}1 {£}{{\pounds}}1
}

\title{TS229}
\begin{document}

%----------- Informations du rapport ---------
\logo{images/logo_em.jpg}
\unif{École nationale supérieure d'électronique, informatique, télécommunications, mathématiques et mécanique de Bordeaux}
\titre{Lecture de code-barres par lancers aléatoires de rayons}
\cours{Département Télécom} %Nom du cours
\sujet{TS225 : Projet Images} %Nom du sujet
\enseignant{Marc \textsc{DONIAS}}
\eleves{Elisa \textsc{Chien} \\
		Alban \textsc{Oberti} \\
		Tierno-Alpha \textsc{Tall} \\
		David \textsc{Yan} 
		}

%----------------------------------------------------------------------------------------
%   TITLE PAGE
%----------------------------------------------------------------------------------------
\fairemarges %Afficher les marges
\fairepagedegarde %Créer la page de garde
\newpage
\tabledematieres %Créer la table de matières
\newpage
% ------------------------------------------------ %

\section{Introduction}

\section {Phase 1 : Détection du code barre}



\section{Phase 2 : Lecture du code barre}



\section{Mise en commun des deux phases}

\section{Bilan de l'organisation}

\subsection{Séance 1}

\begin{table}[h]
	\centering 
	\begin{tabular}{c|c|c}
		& Tâches entreprises& Temps passé\\ \hline
		Elisa& Prise de connaissance du sujet et commencement de la méthode d'Otsu& 2h\\ \hline
		Alban& & \\ \hline
		Tierno& & \\ \hline
		David& & 
	\end{tabular}
	\caption{Organisation de la séance 1}
\end{table}

\subsection{Séance 2}

\begin{table}[h]
	\centering 
	\begin{tabular}{c|c|c}
		& Tâches entreprises& Temps passé\\ \hline
		Elisa& Finition de la méthode d'Otsu & 2h\\ \hline
		Alban& & \\ \hline
		Tierno& & \\ \hline
		David& & 
	\end{tabular}
	\caption{Organisation de la séance 2}
\end{table}

\subsection{Séance 3}

\begin{table}[h]
	\centering 
	\begin{tabular}{c|c|c}
		& Tâches entreprises& Temps passé\\ \hline
		Elisa& Débuggage et mise en commun de toutes les fonctions de la phase 1 & 3h\\ \hline
		Alban& & \\ \hline
		Tierno& & \\ \hline
		David& & 
	\end{tabular}
	\caption{Organisation de la séance 3}
\end{table}

\subsection{Séance 4}

\begin{table}[h]
	\centering 
	\begin{tabular}{c|c|c}
		& Tâches entreprises& Temps passé\\ \hline
		Elisa& & \\ \hline
		Alban& & \\ \hline
		Tierno& & \\ \hline
		David& & 
	\end{tabular}
	\caption{Organisation de la séance 4}
\end{table}

\subsection{Séance 5} % 13 Décembre 

\newpage

\section{Annexes}

\lstset{style=mystyle}

\subsection{Méthode d'Otsu}
\begin{multicols}{2}
	\begin{lstlisting}
		def otsu(img, bins=255, displayHisto=False):
		luminance = None
		
		# Si l'image est en couleur (3 dimensions)
		if len(img.shape) == 3 and img.shape[2] > 1:
			# Calcul de la luminance 
			luminance = np.array([[(img[i][j][0] + img[i][j][1] + img[i][j][2])//3 for j in range(img.shape[1])] for i in range(img.shape[0])])
			luminance = luminance.ravel()
		else:
			luminance = img.ravel()
		
		# Création de l'histogramme
		histogram, bin_edges = np.histogram(luminance.ravel(), range=(0, 255), bins=bins)
		
		# Moyenner l'histogramme
		histogram = histogram/sum(histogram)    
		
		# Création d'un dico pour associer chaque valeur de luminance à sa fréquence
		histogram_dic = {int(bin_edges[i]): histogram[i] for i in range(len(histogram))}
		
		# Initialisation des moyennes et poids initiaux
		n = len(histogram_dic)
		sumB = 0
		wB = 0
		maximum = 0.0
		sum1 = sum(i * histogram_dic[i] for i in range(n))
		total = sum(histogram_dic.values())
		level = 0
		for k in range(1, n):
			wF = total - wB
			if wB > 0 and wF > 0:
				mF = (sum1 - sumB) / wF
				val = wB * wF * (sumB / wB - mF) * (sumB / wB - mF)
				
				if val >= maximum:
					level = k
					maximum = val
			
			wB += histogram_dic[k]
			sumB += (k-1) * histogram_dic[k]
		
		# Afficher l'histogramme
		if displayHisto:
			plt.figure()
			plt.hist(luminance, bins=bins, range=(0, 255))
			plt.axvline(level, color='r')
			plt.title("Histogramme de la luminance")
			plt.xlabel("Luminance")
			plt.ylabel("Fréquence")
			plt.show()
		
		return level
	
	\end{lstlisting}
\end{multicols}

\end{document}