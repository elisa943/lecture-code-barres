%----------------------------------------------------------------------------------------
%  PACKAGES AND CONFIGURATION
%----------------------------------------------------------------------------------------

\documentclass{rapport}
\usepackage{geometry}
\usepackage{fancyhdr} % For custom headers
\usepackage{lastpage} % To determine the last page for the footer
\usepackage{float}
\usepackage{extramarks} % For headers and footers
\usepackage[most]{tcolorbox} % For problem answer sections
\usepackage{graphicx} % For inserting images
\usepackage{xcolor} % For link coloring
\usepackage[hidelinks]{hyperref} % For URL links (no box or color name) 

%Ce que moi(Alban) ai rajouté comme package
\usepackage{amsfonts} % For math sign like Z
\usepackage{amsmath} % For \genfrac
\usepackage{amssymb} % pour \lessgtr
\usepackage{comment} 

\title{TS229}
\begin{document}

%----------- Informations du rapport ---------
\logo{images/logo_em.jpg}
\unif{École nationale supérieure d'électronique, informatique, télécommunications, mathématiques et mécanique de Bordeaux}
\titre{Lecture de code-barres par lancers aléatoires de rayons}
\cours{Département Télécom} %Nom du cours
\sujet{TS225 : Projet Images} %Nom du sujet
\enseignant{Marc \textsc{DONIAS}}
\eleves{Elisa \textsc{Chien} \\
		Alban \textsc{Oberti} \\
		Tierno-Alpha \textsc{Tall} \\
		David \textsc{Yan} 
		}

%----------------------------------------------------------------------------------------
%   TITLE PAGE
%----------------------------------------------------------------------------------------
\fairemarges %Afficher les marges
\fairepagedegarde %Créer la page de garde
\newpage
\tabledematieres %Créer la table de matières
\newpage
% ------------------------------------------------ %

\section{Introduction}

\section{Bilan de l'organisation}

\subsection{Séance 1}

\begin{table}[h]
	\centering 
	\begin{tabular}{c|c|c}
		& Tâches entreprises& Temps passé\\ \hline
		Elisa& Prise de connaissance du sujet et commencement de la méthode d'Otsu& 2h\\ \hline
		Alban& & \\ \hline
		Tierno& & \\ \hline
		David& & 
	\end{tabular}
	\caption{Organisation de la séance 1}
\end{table}

\subsection{Séance 2}

\begin{table}[h]
	\centering 
	\begin{tabular}{c|c|c}
		& Tâches entreprises& Temps passé\\ \hline
		Elisa& Finition de la méthode d'Otsu & 2h\\ \hline
		Alban& & \\ \hline
		Tierno& & \\ \hline
		David& & 
	\end{tabular}
	\caption{Organisation de la séance 2}
\end{table}

\subsection{Séance 3}

\begin{table}[h]
	\centering 
	\begin{tabular}{c|c|c}
		& Tâches entreprises& Temps passé\\ \hline
		Elisa& & \\ \hline
		Alban& & \\ \hline
		Tierno& & \\ \hline
		David& & 
	\end{tabular}
	\caption{Organisation de la séance 3}
\end{table}

\subsection{Séance 4}

\subsection{Séance 5} % 13 Décembre 

\section{Annexes}

\end{document}